\begin{spacing}{1.3}
	
\chapter{Active power loss allocation in distribution network in presence of Distributed Generators}% top level followed by section, subsection
%: ----------------------- contents from here ------------------------
\section{Introduction}
Modern power systems are in continuous transition due to deregulation of the power supply. The deregulation not only brings numerous challenges in the field of power system operation rather shows a significant impact on the electricity markets as well. Traditionally, entire power systems were operated under the state authorities where generation, transmission and distribution companies were own by the single utility. This is known as vertical structure of the power system. In such a structure, consumers have no choice for their services and also, authorities have monopoly in the economical transactions. Therefore, to overcome the drawbacks associated with vertical structure of power system, deregulation has been introduced. Promotions of deregulation unbundle all three power companies, and demolish the monopoly of state authorities. In addition to this, deregulation provides competitive environment for various service providers, so consumers can get the electricity at competitive price. 
\end{spacing}